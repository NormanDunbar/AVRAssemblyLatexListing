 %%%%%%%%%%%%%%%%%%%%%%%%%%%%%%%%%%%%%%%%%%%%%%%%%%%%%%%%%%%%%%%%%%%%%%%%%%%%%%%%
%%% ~ Arduino Language - Arduino IDE Colors ~                                  %%%
%%%                                                                            %%%
%%% Norman Dunbar | 17/03/2023 | MIT Licence                                   %%%
%%% -------------------------------------------------------------------------- %%%
%%% For LaTeX projects:                                                        %%%
%%%                                                                            %%%
%%% Place this file in your source directory where the latex file you're       %%%
%%% creating/editing lives.  Then, add:                                        %%%
%%%     %%%%%%%%%%%%%%%%%%%%%%%%%%%%%%%%%%%%%%%%%%%%%%%%%%%%%%%%%%%%%%%%%%%%%%%%%%%%%%%% 
%%% ~ Arduino Language - Arduino IDE Colors ~                                  %%%
%%%                                                                            %%%
%%% Norman Dunbar | 17/03/2023 | MIT Licence                                   %%%
%%% -------------------------------------------------------------------------- %%%
%%% For LaTeX projects:                                                        %%%
%%%                                                                            %%%
%%% Place this file in your source directory where the latex file you're       %%%
%%% creating/editing lives.  Then, add:                                        %%%
%%%     %%%%%%%%%%%%%%%%%%%%%%%%%%%%%%%%%%%%%%%%%%%%%%%%%%%%%%%%%%%%%%%%%%%%%%%%%%%%%%%% 
%%% ~ Arduino Language - Arduino IDE Colors ~                                  %%%
%%%                                                                            %%%
%%% Norman Dunbar | 17/03/2023 | MIT Licence                                   %%%
%%% -------------------------------------------------------------------------- %%%
%%% For LaTeX projects:                                                        %%%
%%%                                                                            %%%
%%% Place this file in your source directory where the latex file you're       %%%
%%% creating/editing lives.  Then, add:                                        %%%
%%%     %%%%%%%%%%%%%%%%%%%%%%%%%%%%%%%%%%%%%%%%%%%%%%%%%%%%%%%%%%%%%%%%%%%%%%%%%%%%%%%% 
%%% ~ Arduino Language - Arduino IDE Colors ~                                  %%%
%%%                                                                            %%%
%%% Norman Dunbar | 17/03/2023 | MIT Licence                                   %%%
%%% -------------------------------------------------------------------------- %%%
%%% For LaTeX projects:                                                        %%%
%%%                                                                            %%%
%%% Place this file in your source directory where the latex file you're       %%%
%%% creating/editing lives.  Then, add:                                        %%%
%%%    \input{AVRAssemblyLanguage.tex}                                         %%%
%%% in the preamble, before \begin{document}, in your latex file.              %%%
%%%                                                                            %%%
%%% In your document, place your arduino code between:                         %%%
%%%   \begin{lstlisting}[language=AVRAssembly]                                 %%%
%%% and:                                                                       %%%
%%%   \end{lstlisting}                                                         %%%
%%%                                                                            %%%
%%% Or create your own style to add non-built-in functions and variables.      %%%
%%% -------------------------------------------------------------------------- %%%
%%% For LyX projects:                                                          %%%
%%%                                                                            %%%
%%% In Lyx, go to Document->Settings and click on LaTeX Preamble.              %%%
%%%                                                                            %%%
%%% Add the following text to the existing preamble:                           %%%
%%%                                                                            %%%
%%% \lstdefinestyle{AVRasm}{                                                   %%%
%%%   language=AVRAssembly,                                                    %%%
%%%   %% Add other words needing highlighting below %%                         %%%
%%%   morekeywords=[1]{},        % [1] -> dark blue - opcodes                  %%%
%%%   morekeywords=[2]{},        % [2] -> blue - registers & bit names         %%%
%%%   morekeywords=[3]{},        % [3] -> orange - assembler directives        %%%
%%%   morekeywords=[4]{},        % [4] -> green - assembler functions          %%% 
%%% }                                                                          %%%
%%%                                                                            %%%
%%% To highlight some AVR assembly code, right-click the program listing       %%%
%%% insert, and select settings. On the advanced tab, add the text:            %%% 
%%%    style=AVRasm                                                            %%%
%%%                                                                            %%%
%%% To highlight AVR assembly code inserts as a child document, right-click    %%%
%%% the inserted file, choose settings. At the bottom of the dialogue, add:    %%%
%%%    style=AVRasm                                                            %%%
%%% in the "More parameter" editbox on the right side.                         %%%
 %%%%%%%%%%%%%%%%%%%%%%%%%%%%%%%%%%%%%%%%%%%%%%%%%%%%%%%%%%%%%%%%%%%%%%%%%%%%%%%% 

\usepackage{color}
\usepackage{listings}    
\usepackage{courier}

%%% Define Custom IDE Colors %%%
\definecolor{arduinoGreen}    {rgb} {0.17, 0.43, 0.01}
\definecolor{arduinoGrey}     {rgb} {0.47, 0.47, 0.33}
\definecolor{arduinoOrange}   {rgb} {0.8 , 0.4 , 0   }
\definecolor{arduinoBlue}     {rgb} {0.01, 0.61, 0.98}
\definecolor{arduinoDarkBlue} {rgb} {0.0 , 0.2 , 0.5 }

%%% Define AVR Assembly Language %%%
\lstdefinelanguage{AVRAssembly}{
  sensitive=false,    % Keywords etc are not case sensitive
  morecomment=[l]{;}, % Line comments begin with ';'
  morestring=[b]",    % Strings are in double quotes
%
%
  %%% Keyword Color Group 1 %%%  (called KEYWORD3 by arduino)
  keywordstyle=\color{arduinoDarkBlue},   
  morekeywords={   % Add assembly instructions to group 1
                ADC, ADD, ADIW, AND, ANDI, ASR,
                %
                BCLR, BLD, BRBC, BRBS, BRCC, BRCS, BREAK,
                BREQ, BRGE, BRHC, BRID, BRIE, BRLO, BRLT, 
                BRMI, BRNE, BRPL, BRSH, BRTC, BRTS, BRVC, BRVS,
                BSET, BST,
                %
                CALL, CBI, CBR, CLC, CLH, CLI, CLN, CLR, CLS,
                CLT, CLV, CLZ, COM, CP, CPC, CPI, CPSE,
                %
                DEC, DES,
                %
                EICALL, EIJMP, ELPM, EOR,
                %
                FMUL, FMULS, FMULSU,
                %
                ICALL, IJMP, IN, INC,
                %
                JMP,
                %
                LAC, LAS, LAT, LD, LDD, LDI, LDS, LPM, LSL, 
                LSR, 
                %
                MOV, MOVW, MUL, MULS, MULSU,
                %
                NEG, NOP,
                %
                OR, ORI, OUT,
                %
                POP, PUSH,
                %
                RCALL, RET, RETI, RJMP, ROL, ROR, 
                %
                SBC, SBCI, SBI, SBIC, SBIS, SBIW, SBR, SBRC,
                SBRS, SEC, SEH, SEI, SEN, SER, SES, SET, SEV, 
                SEZ, SLEEP, SPM, ST, STD, STS, SUB, SUBI, SWAP,
                %
                TST, 
                %
                WDR, 
                %
                XCH,
                }, 
% 
%
  %%% Keyword Color Group 2 %%%  (called LITERAL1 by arduino)
  keywordstyle=[2]\color{arduinoBlue},   
  keywords=[2]{ % Add registers and bit names to 2nd group  
                ACBG, ACD, ACI, ACIC, ACIE, ACIS0, ACIS1, ACME, 
                ACO, ACSR, ADATE, ADC, ADC0D, ADC1D, ADC2D, 
                ADC3D, ADC4D, ADC5D, ADCH, ADCL, ADCSRA, ADCSRB, 
                ADCW, ADEN, ADIE, ADIF, ADLAR, ADMUX, ADPS0, 
                ADPS1, ADPS2, ADSC, ADTS0, ADTS1, ADTS2, AIN0D, 
                AIN1D, AS2, ASSR, 
                %
                BLBSET, BODS, BODSE, BORF, 
                %
                CLKPCE, CLKPR, CLKPS0, CLKPS1, CLKPS2, CLKPS3, 
                COM0A0, COM0A1, COM0B0, COM0B1, COM1A0, COM1A1, 
                COM1B0, COM1B1, COM2A0, COM2A1, COM2B0, COM2B1, 
                CPHA, CPOL, CS00, CS01, CS02, CS10, CS11, CS12, 
                CS20, CS21, CS22, 
                %
                DDB0, DDB1, DDB2, DDB3, DDB4, DDB5, DDB6, DDB7, 
                DDC0, DDC1, DDC2, DDC3, DDC4, DDC5, DDC6, 
                DDD0, DDD1, DDD2, DDD3, DDD4, DDD5, DDD6, DDD7, 
                DDRB, DDRC, DDRD, DIDR0, DIDR1, DOR0, DORD, 
                %
                EEAR, EEARH, EEARL, EECR, EEDR, EEMPE, EEPE, 
                EEPM0, EEPM1, EERE, EERIE, EICRA, EIFR, EIMSK, 
                EXCLK, EXTRF, FE0, 
                %
                FOC0A, FOC0B, FOC1A, FOC1B, FOC2A, FOC2B, 
                %
                GPIOR0, GPIOR1, GPIOR2, GTCCR, 
                %
                ICES1, ICF1, ICIE1, ICNC1, ICR1, ICR1H, ICR1L, 
                IDRIE0, INT0, INT1, INTF0, INTF1, ISC00, ISC01, 
                ISC10, ISC11, IVCE, IVSEL, 
                %
                MCUCR, MCUSR, MPCM0, MSTR, MUX0, MUX1, MUX2, 
                MUX3, 
                %
                OCF0A, OCF0B, OCF1A, OCF1B, OCF2A, OCF2B, 
                OCIE0A, OCIE0B, OCIE1A, OCIE1B, OCIE2A, OCIE2B,
                OCR0A, OCR0B, OCR1A, OCR1AH, OCR1AL, OCR1B, 
                OCR1BH, OCR1BL, OCR2A, OCR2AUB, OCR2B, OCR2BUB,
                OSCCAL, 
                %
                PCICR, PCIE0, PCIE1, PCIE2, PCIF0, PCIF1, 
                PCIF2, PCIFR, 
                %
                PCINT0, PCINT1, PCINT2, PCINT3, PCINT4, PCINT6,
                PCINT7, PCINT8, PCINT9, PCINT10, PCINT11, 
                PCINT12, PCINT13, PCINT14, PCINT16, PCINT17, 
                PCINT18, PCINT19, PCINT20, PCINT22, PCINT23, 
                %
                PCMSK0, PCMSK1, PCMSK2, PGERS, PGWRT, 
                %
                PINB, 
                PINB0, PINB1, PINB2, PINB3, PINB4, PINB5, PINB6,
                PINB7, 
                PINC, 
                PINC0, PINC1, PINC2, PINC3, PINC4, PINC5, PINC6,  
                PIND, 
                PIND0, PIND1, PIND2, PIND3, PIND4, PIND5, PIND6,
                PIND7, 
                %
                PORF, 
                PORTB, 
                PORTB0, PORTB1, PORTB2, PORTB3, PORTB4, PORTB5,
                PORTB6, PORTB7, 
                PORTC, 
                PORTC0, PORTC1, PORTC2, PORTC3, PORTC4, PORTC5,
                PORTC6, 
                PORTD, 
                PORTD0, PORTD1, PORTD2, PORTD3, PORTD4, PORTD5,
                PORTD6, PORTD7, 
                %
                PRADC, PRR, PRSPI, PRTIM0, PRTIM1, PRTIM2, 
                PRTWI, PRUSART0, PSRASY, PSRSYNC, PUD, PVINT21,
                PVINT5, 
                %
                REFS0, REFS1, RWWSB, RWWSRE, RXB80, RXC0, 
                RXCIE0, RXEN0, 
                %
                SE, SELFPRGEN, SM0, SM1, SM2, SMCR, SP0, SP1, 
                SP10, 
                SP2, SP3, SP4, SP5, SP6, SP7, SP8, SP9, 
                SPCR, SPDR, SPE, SPH, SPI2X, SPIE, SPIF, SPL,
                SPMCSR, SPMIE, SPR0, SPR1, SPSR, SREG, 
                %
                TCCR1A, TCCR1B, TCCR1C, TCCR2A, TCCR2B, TCN2UB,
                TCNT0, TCNT1, TCNT1H, TCNT1L, TCNT2, TCR2AUB, 
                TCR2BUB, TIFR0, TIFR1, TIFR2, TIMSK0, TIMSK1,
                TIMSK2, TOIE0, TOIE1, TOIE2, TOV0, TOV1, TOV2, 
                TSM, TTCR0A, TTCR0B, TWA0, TWA1, TWA2, TWA3, 
                TWA4, TWA5, TWA6, TWAM0, TWAM1, TWAM2, TWAM3, 
                TWAM4, TWAM5, TWAM6, TWAMR, TWAR, TWBR, TWCR, 
                TWDR, TWEA, TWEN, TWGCE, TWIE, TWINT, 
                TWPS0, TWPS1, TWS3, TWS4, TWS5, TWS6, TWS7, 
                TWSR, TWSTA, TWSTO, TWWC, TXB80, TXC0, TXCIE0, 
                TXEN0, 
                %
                U2X0, UBRR0, UBRR0H, UBRR0L, UCPHA0, UCPOL0, 
                UCSR0A, UCSR0B, UCSR0C, UCSZ00, UCSZ01, UCSZ02,
                UDORD0, UDR0, UDRE0, UMSEL00, UMSEL01, UPE0, 
                UPM00, UPM01, USBS0, 
                %
                WCOL, WDCE, WDE, WDIE, WDIF, WDP0, WDP1, WDP2, 
                WDP3, WDRF, WDTCSR, WGM00, WGM01, WGM02, WGM10,
                WGM11, WGM12, WGM13, WGM20, WGM21, WGM22, 
                %
                % Fuses
                %
                BODLEVEL0, BODLEVEL1, BODLEVEL2, 
                BOOTRST, BOOTSZ0, BOOTSZ1, 
                CKDIV8, CKOUT, 
                CKSEL0, CKSEL1, CKSEL2, CKSEL3, 
                DWEN, EESAVE, RSTDISBL, SPOIEN, 
                SUT0, SUT1, WDTON, 
                %
                % Lock Bits
                %
                BLB01, BLB02, BLB11, BLB12,
                LB1, LB2
                %
                % General purpose registers
                %
                R0,  R1,  R2,  R3,  R4,  R5,  R6,  R7,  R8, 
                R9,  R10, R11, R12, R13, R14, R15, R16, R17, 
                R18, R19, R20, R21, R22, R23, R24, R25, R26, 
                R27, R28, R29, R30, R31, 
                X, XH, XL, Y, YH, YL, Z, ZH, ZL,
                },  
% 
%
  %%% Keyword Color Group 3 %%%  (called KEYWORD1 by arduino)
  keywordstyle=[3]\color{arduinoOrange},
  keywords=[3]{  % Add AVR assembler directives as a 3rd group
                BYTE,
                %
                CSEG, CSEGSIZE,
                %
                DB, DD, DEF, DEVICE, DSEG, DQ, DW,
                %
                ELSE, ELIF, ENDIF, ENDM, ENDMACRO, EQU, 
                ERROR, ESEG, EXIT,
                %
                IF, IFDEF, IFNDEF, INCLUDE,
                %
                LIST, LISTMAC,
                %
                MACRO, MESSAGE,
                %
                NOLIST, NOOVERLAP,
                %
                ORG, OVERLAP,
                %
                SET,
                %
                UNDEF, 
                %
                % Add AVR-GAS assembler directives
                %
                ABORT, ALIGN, APP-FILE, ASCII, ASCIZ, 
                %
                BALIGN, BYTE, 
                %
                COMM, 
                %
                DATA, DEF, DESC, DIM, DOUBLE, 
                %
                EJECT, ELSE, ENDEF, ENDIF, EQU, EXTERN, 
                %
                FILE, FILL, FLOAT, 
                %
                GLOBAL, 
                %
                HWORD,
                %
                IDENT, IF, INCLUDE, INT, IRP, IRPC, 
                %
                LCOMM, LFLAGS, LINE, LN, LIST, 
                %
                MACRO,
                %
                NOLIST, 
                %
                OCTA, ORG, 
                %
                P2ALIGN, PSIZE, 
                %
                QUAD, 
                %
                REPT, 
                %
                SUBTTL, SCL, SECTION, SET, SHORT, 
                SINGLE, SHORT, SPACE, STABD, STABN, 
                STABS, STRING, 
                %
                TAG, TEXT, TITLE, TYPE, 
                %
                VAL, 
                %
                WORD,
                %
                % Add AVR-GCC preprocessor directives
                %
                DEFINE, INCLUDE, UNDEF, WARNING, PRAGMA, 
                }, 
%
%
  %%% Keyword Color Group 4 %%%  (called KEYWORD2 by arduino)
  keywordstyle=[4]\color{arduinoGreen},
  keywords=[4]{  % Add AVR assembler functions as a 4th group
                ABS, 
                BYTE2, BYTE3, BYTE4, 
                DEFINED,
                EXP2,
                FRAC, 
                HIGH, HWRD, 
                INT,
                LOW, LWRD, LOG2, 
                PAGE, 
                Q7, Q15, 
                STRLEN,
                %
                % And some macros from avrlib
                %
                __SFR_OFFSET, _SFR_IO_ADDR, _SFR_IO_REG_P,
                _SFR_MEM_ADDR, _SFR_ASM_COMPAT, _SFR_MEM8,
                _SFR_MEM16, _SFR_MEM32, _SFR_IO8, _SFR_IO16,
                _SFR_BYTE, _SFR_WORD, _SFR_DWORD
                %
                % And avr-gas functions.
                %
                HH8, HI8, HLO8, HHI8,
                LO8,
                PM, PM_HH8, PM_HI8, PM_LO8,
                },      
%
%
  %%% Set Other Colors %%%
  stringstyle=\color{arduinoDarkBlue},    
  commentstyle=\color{arduinoGrey},    
%          
%   
  %%%% Line Numbering %%%%
  numbers=left,                    
  numbersep=5pt,                   
  numberstyle=\color{arduinoGrey}, 
  showstringspaces=false
  showlines=true
  emptylines=1
  numbers=left   
  %stepnumber=2,                      % show every 2 line numbers
%
%
  %%%% Code Box Style %%%%
  breaklines=true,                    % wordwrapping
  tabsize=4,         
  basicstyle=\ttfamily  
}
                                         %%%
%%% in the preamble, before \begin{document}, in your latex file.              %%%
%%%                                                                            %%%
%%% In your document, place your arduino code between:                         %%%
%%%   \begin{lstlisting}[language=AVRAssembly]                                 %%%
%%% and:                                                                       %%%
%%%   \end{lstlisting}                                                         %%%
%%%                                                                            %%%
%%% Or create your own style to add non-built-in functions and variables.      %%%
%%% -------------------------------------------------------------------------- %%%
%%% For LyX projects:                                                          %%%
%%%                                                                            %%%
%%% In Lyx, go to Document->Settings and click on LaTeX Preamble.              %%%
%%%                                                                            %%%
%%% Add the following text to the existing preamble:                           %%%
%%%                                                                            %%%
%%% \lstdefinestyle{AVRasm}{                                                   %%%
%%%   language=AVRAssembly,                                                    %%%
%%%   %% Add other words needing highlighting below %%                         %%%
%%%   morekeywords=[1]{},        % [1] -> dark blue - opcodes                  %%%
%%%   morekeywords=[2]{},        % [2] -> blue - registers & bit names         %%%
%%%   morekeywords=[3]{},        % [3] -> orange - assembler directives        %%%
%%%   morekeywords=[4]{},        % [4] -> green - assembler functions          %%% 
%%% }                                                                          %%%
%%%                                                                            %%%
%%% To highlight some AVR assembly code, right-click the program listing       %%%
%%% insert, and select settings. On the advanced tab, add the text:            %%% 
%%%    style=AVRasm                                                            %%%
%%%                                                                            %%%
%%% To highlight AVR assembly code inserts as a child document, right-click    %%%
%%% the inserted file, choose settings. At the bottom of the dialogue, add:    %%%
%%%    style=AVRasm                                                            %%%
%%% in the "More parameter" editbox on the right side.                         %%%
 %%%%%%%%%%%%%%%%%%%%%%%%%%%%%%%%%%%%%%%%%%%%%%%%%%%%%%%%%%%%%%%%%%%%%%%%%%%%%%%% 

\usepackage{color}
\usepackage{listings}    
\usepackage{courier}

%%% Define Custom IDE Colors %%%
\definecolor{arduinoGreen}    {rgb} {0.17, 0.43, 0.01}
\definecolor{arduinoGrey}     {rgb} {0.47, 0.47, 0.33}
\definecolor{arduinoOrange}   {rgb} {0.8 , 0.4 , 0   }
\definecolor{arduinoBlue}     {rgb} {0.01, 0.61, 0.98}
\definecolor{arduinoDarkBlue} {rgb} {0.0 , 0.2 , 0.5 }

%%% Define AVR Assembly Language %%%
\lstdefinelanguage{AVRAssembly}{
  sensitive=false,    % Keywords etc are not case sensitive
  morecomment=[l]{;}, % Line comments begin with ';'
  morestring=[b]",    % Strings are in double quotes
%
%
  %%% Keyword Color Group 1 %%%  (called KEYWORD3 by arduino)
  keywordstyle=\color{arduinoDarkBlue},   
  morekeywords={   % Add assembly instructions to group 1
                ADC, ADD, ADIW, AND, ANDI, ASR,
                %
                BCLR, BLD, BRBC, BRBS, BRCC, BRCS, BREAK,
                BREQ, BRGE, BRHC, BRID, BRIE, BRLO, BRLT, 
                BRMI, BRNE, BRPL, BRSH, BRTC, BRTS, BRVC, BRVS,
                BSET, BST,
                %
                CALL, CBI, CBR, CLC, CLH, CLI, CLN, CLR, CLS,
                CLT, CLV, CLZ, COM, CP, CPC, CPI, CPSE,
                %
                DEC, DES,
                %
                EICALL, EIJMP, ELPM, EOR,
                %
                FMUL, FMULS, FMULSU,
                %
                ICALL, IJMP, IN, INC,
                %
                JMP,
                %
                LAC, LAS, LAT, LD, LDD, LDI, LDS, LPM, LSL, 
                LSR, 
                %
                MOV, MOVW, MUL, MULS, MULSU,
                %
                NEG, NOP,
                %
                OR, ORI, OUT,
                %
                POP, PUSH,
                %
                RCALL, RET, RETI, RJMP, ROL, ROR, 
                %
                SBC, SBCI, SBI, SBIC, SBIS, SBIW, SBR, SBRC,
                SBRS, SEC, SEH, SEI, SEN, SER, SES, SET, SEV, 
                SEZ, SLEEP, SPM, ST, STD, STS, SUB, SUBI, SWAP,
                %
                TST, 
                %
                WDR, 
                %
                XCH,
                }, 
% 
%
  %%% Keyword Color Group 2 %%%  (called LITERAL1 by arduino)
  keywordstyle=[2]\color{arduinoBlue},   
  keywords=[2]{ % Add registers and bit names to 2nd group  
                ACBG, ACD, ACI, ACIC, ACIE, ACIS0, ACIS1, ACME, 
                ACO, ACSR, ADATE, ADC, ADC0D, ADC1D, ADC2D, 
                ADC3D, ADC4D, ADC5D, ADCH, ADCL, ADCSRA, ADCSRB, 
                ADCW, ADEN, ADIE, ADIF, ADLAR, ADMUX, ADPS0, 
                ADPS1, ADPS2, ADSC, ADTS0, ADTS1, ADTS2, AIN0D, 
                AIN1D, AS2, ASSR, 
                %
                BLBSET, BODS, BODSE, BORF, 
                %
                CLKPCE, CLKPR, CLKPS0, CLKPS1, CLKPS2, CLKPS3, 
                COM0A0, COM0A1, COM0B0, COM0B1, COM1A0, COM1A1, 
                COM1B0, COM1B1, COM2A0, COM2A1, COM2B0, COM2B1, 
                CPHA, CPOL, CS00, CS01, CS02, CS10, CS11, CS12, 
                CS20, CS21, CS22, 
                %
                DDB0, DDB1, DDB2, DDB3, DDB4, DDB5, DDB6, DDB7, 
                DDC0, DDC1, DDC2, DDC3, DDC4, DDC5, DDC6, 
                DDD0, DDD1, DDD2, DDD3, DDD4, DDD5, DDD6, DDD7, 
                DDRB, DDRC, DDRD, DIDR0, DIDR1, DOR0, DORD, 
                %
                EEAR, EEARH, EEARL, EECR, EEDR, EEMPE, EEPE, 
                EEPM0, EEPM1, EERE, EERIE, EICRA, EIFR, EIMSK, 
                EXCLK, EXTRF, FE0, 
                %
                FOC0A, FOC0B, FOC1A, FOC1B, FOC2A, FOC2B, 
                %
                GPIOR0, GPIOR1, GPIOR2, GTCCR, 
                %
                ICES1, ICF1, ICIE1, ICNC1, ICR1, ICR1H, ICR1L, 
                IDRIE0, INT0, INT1, INTF0, INTF1, ISC00, ISC01, 
                ISC10, ISC11, IVCE, IVSEL, 
                %
                MCUCR, MCUSR, MPCM0, MSTR, MUX0, MUX1, MUX2, 
                MUX3, 
                %
                OCF0A, OCF0B, OCF1A, OCF1B, OCF2A, OCF2B, 
                OCIE0A, OCIE0B, OCIE1A, OCIE1B, OCIE2A, OCIE2B,
                OCR0A, OCR0B, OCR1A, OCR1AH, OCR1AL, OCR1B, 
                OCR1BH, OCR1BL, OCR2A, OCR2AUB, OCR2B, OCR2BUB,
                OSCCAL, 
                %
                PCICR, PCIE0, PCIE1, PCIE2, PCIF0, PCIF1, 
                PCIF2, PCIFR, 
                %
                PCINT0, PCINT1, PCINT2, PCINT3, PCINT4, PCINT6,
                PCINT7, PCINT8, PCINT9, PCINT10, PCINT11, 
                PCINT12, PCINT13, PCINT14, PCINT16, PCINT17, 
                PCINT18, PCINT19, PCINT20, PCINT22, PCINT23, 
                %
                PCMSK0, PCMSK1, PCMSK2, PGERS, PGWRT, 
                %
                PINB, 
                PINB0, PINB1, PINB2, PINB3, PINB4, PINB5, PINB6,
                PINB7, 
                PINC, 
                PINC0, PINC1, PINC2, PINC3, PINC4, PINC5, PINC6,  
                PIND, 
                PIND0, PIND1, PIND2, PIND3, PIND4, PIND5, PIND6,
                PIND7, 
                %
                PORF, 
                PORTB, 
                PORTB0, PORTB1, PORTB2, PORTB3, PORTB4, PORTB5,
                PORTB6, PORTB7, 
                PORTC, 
                PORTC0, PORTC1, PORTC2, PORTC3, PORTC4, PORTC5,
                PORTC6, 
                PORTD, 
                PORTD0, PORTD1, PORTD2, PORTD3, PORTD4, PORTD5,
                PORTD6, PORTD7, 
                %
                PRADC, PRR, PRSPI, PRTIM0, PRTIM1, PRTIM2, 
                PRTWI, PRUSART0, PSRASY, PSRSYNC, PUD, PVINT21,
                PVINT5, 
                %
                REFS0, REFS1, RWWSB, RWWSRE, RXB80, RXC0, 
                RXCIE0, RXEN0, 
                %
                SE, SELFPRGEN, SM0, SM1, SM2, SMCR, SP0, SP1, 
                SP10, 
                SP2, SP3, SP4, SP5, SP6, SP7, SP8, SP9, 
                SPCR, SPDR, SPE, SPH, SPI2X, SPIE, SPIF, SPL,
                SPMCSR, SPMIE, SPR0, SPR1, SPSR, SREG, 
                %
                TCCR1A, TCCR1B, TCCR1C, TCCR2A, TCCR2B, TCN2UB,
                TCNT0, TCNT1, TCNT1H, TCNT1L, TCNT2, TCR2AUB, 
                TCR2BUB, TIFR0, TIFR1, TIFR2, TIMSK0, TIMSK1,
                TIMSK2, TOIE0, TOIE1, TOIE2, TOV0, TOV1, TOV2, 
                TSM, TTCR0A, TTCR0B, TWA0, TWA1, TWA2, TWA3, 
                TWA4, TWA5, TWA6, TWAM0, TWAM1, TWAM2, TWAM3, 
                TWAM4, TWAM5, TWAM6, TWAMR, TWAR, TWBR, TWCR, 
                TWDR, TWEA, TWEN, TWGCE, TWIE, TWINT, 
                TWPS0, TWPS1, TWS3, TWS4, TWS5, TWS6, TWS7, 
                TWSR, TWSTA, TWSTO, TWWC, TXB80, TXC0, TXCIE0, 
                TXEN0, 
                %
                U2X0, UBRR0, UBRR0H, UBRR0L, UCPHA0, UCPOL0, 
                UCSR0A, UCSR0B, UCSR0C, UCSZ00, UCSZ01, UCSZ02,
                UDORD0, UDR0, UDRE0, UMSEL00, UMSEL01, UPE0, 
                UPM00, UPM01, USBS0, 
                %
                WCOL, WDCE, WDE, WDIE, WDIF, WDP0, WDP1, WDP2, 
                WDP3, WDRF, WDTCSR, WGM00, WGM01, WGM02, WGM10,
                WGM11, WGM12, WGM13, WGM20, WGM21, WGM22, 
                %
                % Fuses
                %
                BODLEVEL0, BODLEVEL1, BODLEVEL2, 
                BOOTRST, BOOTSZ0, BOOTSZ1, 
                CKDIV8, CKOUT, 
                CKSEL0, CKSEL1, CKSEL2, CKSEL3, 
                DWEN, EESAVE, RSTDISBL, SPOIEN, 
                SUT0, SUT1, WDTON, 
                %
                % Lock Bits
                %
                BLB01, BLB02, BLB11, BLB12,
                LB1, LB2
                %
                % General purpose registers
                %
                R0,  R1,  R2,  R3,  R4,  R5,  R6,  R7,  R8, 
                R9,  R10, R11, R12, R13, R14, R15, R16, R17, 
                R18, R19, R20, R21, R22, R23, R24, R25, R26, 
                R27, R28, R29, R30, R31, 
                X, XH, XL, Y, YH, YL, Z, ZH, ZL,
                },  
% 
%
  %%% Keyword Color Group 3 %%%  (called KEYWORD1 by arduino)
  keywordstyle=[3]\color{arduinoOrange},
  keywords=[3]{  % Add AVR assembler directives as a 3rd group
                BYTE,
                %
                CSEG, CSEGSIZE,
                %
                DB, DD, DEF, DEVICE, DSEG, DQ, DW,
                %
                ELSE, ELIF, ENDIF, ENDM, ENDMACRO, EQU, 
                ERROR, ESEG, EXIT,
                %
                IF, IFDEF, IFNDEF, INCLUDE,
                %
                LIST, LISTMAC,
                %
                MACRO, MESSAGE,
                %
                NOLIST, NOOVERLAP,
                %
                ORG, OVERLAP,
                %
                SET,
                %
                UNDEF, 
                %
                % Add AVR-GAS assembler directives
                %
                ABORT, ALIGN, APP-FILE, ASCII, ASCIZ, 
                %
                BALIGN, BYTE, 
                %
                COMM, 
                %
                DATA, DEF, DESC, DIM, DOUBLE, 
                %
                EJECT, ELSE, ENDEF, ENDIF, EQU, EXTERN, 
                %
                FILE, FILL, FLOAT, 
                %
                GLOBAL, 
                %
                HWORD,
                %
                IDENT, IF, INCLUDE, INT, IRP, IRPC, 
                %
                LCOMM, LFLAGS, LINE, LN, LIST, 
                %
                MACRO,
                %
                NOLIST, 
                %
                OCTA, ORG, 
                %
                P2ALIGN, PSIZE, 
                %
                QUAD, 
                %
                REPT, 
                %
                SUBTTL, SCL, SECTION, SET, SHORT, 
                SINGLE, SHORT, SPACE, STABD, STABN, 
                STABS, STRING, 
                %
                TAG, TEXT, TITLE, TYPE, 
                %
                VAL, 
                %
                WORD,
                %
                % Add AVR-GCC preprocessor directives
                %
                DEFINE, INCLUDE, UNDEF, WARNING, PRAGMA, 
                }, 
%
%
  %%% Keyword Color Group 4 %%%  (called KEYWORD2 by arduino)
  keywordstyle=[4]\color{arduinoGreen},
  keywords=[4]{  % Add AVR assembler functions as a 4th group
                ABS, 
                BYTE2, BYTE3, BYTE4, 
                DEFINED,
                EXP2,
                FRAC, 
                HIGH, HWRD, 
                INT,
                LOW, LWRD, LOG2, 
                PAGE, 
                Q7, Q15, 
                STRLEN,
                %
                % And some macros from avrlib
                %
                __SFR_OFFSET, _SFR_IO_ADDR, _SFR_IO_REG_P,
                _SFR_MEM_ADDR, _SFR_ASM_COMPAT, _SFR_MEM8,
                _SFR_MEM16, _SFR_MEM32, _SFR_IO8, _SFR_IO16,
                _SFR_BYTE, _SFR_WORD, _SFR_DWORD
                %
                % And avr-gas functions.
                %
                HH8, HI8, HLO8, HHI8,
                LO8,
                PM, PM_HH8, PM_HI8, PM_LO8,
                },      
%
%
  %%% Set Other Colors %%%
  stringstyle=\color{arduinoDarkBlue},    
  commentstyle=\color{arduinoGrey},    
%          
%   
  %%%% Line Numbering %%%%
  numbers=left,                    
  numbersep=5pt,                   
  numberstyle=\color{arduinoGrey}, 
  showstringspaces=false
  showlines=true
  emptylines=1
  numbers=left   
  %stepnumber=2,                      % show every 2 line numbers
%
%
  %%%% Code Box Style %%%%
  breaklines=true,                    % wordwrapping
  tabsize=4,         
  basicstyle=\ttfamily  
}
                                         %%%
%%% in the preamble, before \begin{document}, in your latex file.              %%%
%%%                                                                            %%%
%%% In your document, place your arduino code between:                         %%%
%%%   \begin{lstlisting}[language=AVRAssembly]                                 %%%
%%% and:                                                                       %%%
%%%   \end{lstlisting}                                                         %%%
%%%                                                                            %%%
%%% Or create your own style to add non-built-in functions and variables.      %%%
%%% -------------------------------------------------------------------------- %%%
%%% For LyX projects:                                                          %%%
%%%                                                                            %%%
%%% In Lyx, go to Document->Settings and click on LaTeX Preamble.              %%%
%%%                                                                            %%%
%%% Add the following text to the existing preamble:                           %%%
%%%                                                                            %%%
%%% \lstdefinestyle{AVRasm}{                                                   %%%
%%%   language=AVRAssembly,                                                    %%%
%%%   %% Add other words needing highlighting below %%                         %%%
%%%   morekeywords=[1]{},        % [1] -> dark blue - opcodes                  %%%
%%%   morekeywords=[2]{},        % [2] -> blue - registers & bit names         %%%
%%%   morekeywords=[3]{},        % [3] -> orange - assembler directives        %%%
%%%   morekeywords=[4]{},        % [4] -> green - assembler functions          %%% 
%%% }                                                                          %%%
%%%                                                                            %%%
%%% To highlight some AVR assembly code, right-click the program listing       %%%
%%% insert, and select settings. On the advanced tab, add the text:            %%% 
%%%    style=AVRasm                                                            %%%
%%%                                                                            %%%
%%% To highlight AVR assembly code inserts as a child document, right-click    %%%
%%% the inserted file, choose settings. At the bottom of the dialogue, add:    %%%
%%%    style=AVRasm                                                            %%%
%%% in the "More parameter" editbox on the right side.                         %%%
 %%%%%%%%%%%%%%%%%%%%%%%%%%%%%%%%%%%%%%%%%%%%%%%%%%%%%%%%%%%%%%%%%%%%%%%%%%%%%%%% 

\usepackage{color}
\usepackage{listings}    
\usepackage{courier}

%%% Define Custom IDE Colors %%%
\definecolor{arduinoGreen}    {rgb} {0.17, 0.43, 0.01}
\definecolor{arduinoGrey}     {rgb} {0.47, 0.47, 0.33}
\definecolor{arduinoOrange}   {rgb} {0.8 , 0.4 , 0   }
\definecolor{arduinoBlue}     {rgb} {0.01, 0.61, 0.98}
\definecolor{arduinoDarkBlue} {rgb} {0.0 , 0.2 , 0.5 }

%%% Define AVR Assembly Language %%%
\lstdefinelanguage{AVRAssembly}{
  sensitive=false,    % Keywords etc are not case sensitive
  morecomment=[l]{;}, % Line comments begin with ';'
  morestring=[b]",    % Strings are in double quotes
%
%
  %%% Keyword Color Group 1 %%%  (called KEYWORD3 by arduino)
  keywordstyle=\color{arduinoDarkBlue},   
  morekeywords={   % Add assembly instructions to group 1
                ADC, ADD, ADIW, AND, ANDI, ASR,
                %
                BCLR, BLD, BRBC, BRBS, BRCC, BRCS, BREAK,
                BREQ, BRGE, BRHC, BRID, BRIE, BRLO, BRLT, 
                BRMI, BRNE, BRPL, BRSH, BRTC, BRTS, BRVC, BRVS,
                BSET, BST,
                %
                CALL, CBI, CBR, CLC, CLH, CLI, CLN, CLR, CLS,
                CLT, CLV, CLZ, COM, CP, CPC, CPI, CPSE,
                %
                DEC, DES,
                %
                EICALL, EIJMP, ELPM, EOR,
                %
                FMUL, FMULS, FMULSU,
                %
                ICALL, IJMP, IN, INC,
                %
                JMP,
                %
                LAC, LAS, LAT, LD, LDD, LDI, LDS, LPM, LSL, 
                LSR, 
                %
                MOV, MOVW, MUL, MULS, MULSU,
                %
                NEG, NOP,
                %
                OR, ORI, OUT,
                %
                POP, PUSH,
                %
                RCALL, RET, RETI, RJMP, ROL, ROR, 
                %
                SBC, SBCI, SBI, SBIC, SBIS, SBIW, SBR, SBRC,
                SBRS, SEC, SEH, SEI, SEN, SER, SES, SET, SEV, 
                SEZ, SLEEP, SPM, ST, STD, STS, SUB, SUBI, SWAP,
                %
                TST, 
                %
                WDR, 
                %
                XCH,
                }, 
% 
%
  %%% Keyword Color Group 2 %%%  (called LITERAL1 by arduino)
  keywordstyle=[2]\color{arduinoBlue},   
  keywords=[2]{ % Add registers and bit names to 2nd group  
                ACBG, ACD, ACI, ACIC, ACIE, ACIS0, ACIS1, ACME, 
                ACO, ACSR, ADATE, ADC, ADC0D, ADC1D, ADC2D, 
                ADC3D, ADC4D, ADC5D, ADCH, ADCL, ADCSRA, ADCSRB, 
                ADCW, ADEN, ADIE, ADIF, ADLAR, ADMUX, ADPS0, 
                ADPS1, ADPS2, ADSC, ADTS0, ADTS1, ADTS2, AIN0D, 
                AIN1D, AS2, ASSR, 
                %
                BLBSET, BODS, BODSE, BORF, 
                %
                CLKPCE, CLKPR, CLKPS0, CLKPS1, CLKPS2, CLKPS3, 
                COM0A0, COM0A1, COM0B0, COM0B1, COM1A0, COM1A1, 
                COM1B0, COM1B1, COM2A0, COM2A1, COM2B0, COM2B1, 
                CPHA, CPOL, CS00, CS01, CS02, CS10, CS11, CS12, 
                CS20, CS21, CS22, 
                %
                DDB0, DDB1, DDB2, DDB3, DDB4, DDB5, DDB6, DDB7, 
                DDC0, DDC1, DDC2, DDC3, DDC4, DDC5, DDC6, 
                DDD0, DDD1, DDD2, DDD3, DDD4, DDD5, DDD6, DDD7, 
                DDRB, DDRC, DDRD, DIDR0, DIDR1, DOR0, DORD, 
                %
                EEAR, EEARH, EEARL, EECR, EEDR, EEMPE, EEPE, 
                EEPM0, EEPM1, EERE, EERIE, EICRA, EIFR, EIMSK, 
                EXCLK, EXTRF, FE0, 
                %
                FOC0A, FOC0B, FOC1A, FOC1B, FOC2A, FOC2B, 
                %
                GPIOR0, GPIOR1, GPIOR2, GTCCR, 
                %
                ICES1, ICF1, ICIE1, ICNC1, ICR1, ICR1H, ICR1L, 
                IDRIE0, INT0, INT1, INTF0, INTF1, ISC00, ISC01, 
                ISC10, ISC11, IVCE, IVSEL, 
                %
                MCUCR, MCUSR, MPCM0, MSTR, MUX0, MUX1, MUX2, 
                MUX3, 
                %
                OCF0A, OCF0B, OCF1A, OCF1B, OCF2A, OCF2B, 
                OCIE0A, OCIE0B, OCIE1A, OCIE1B, OCIE2A, OCIE2B,
                OCR0A, OCR0B, OCR1A, OCR1AH, OCR1AL, OCR1B, 
                OCR1BH, OCR1BL, OCR2A, OCR2AUB, OCR2B, OCR2BUB,
                OSCCAL, 
                %
                PCICR, PCIE0, PCIE1, PCIE2, PCIF0, PCIF1, 
                PCIF2, PCIFR, 
                %
                PCINT0, PCINT1, PCINT2, PCINT3, PCINT4, PCINT6,
                PCINT7, PCINT8, PCINT9, PCINT10, PCINT11, 
                PCINT12, PCINT13, PCINT14, PCINT16, PCINT17, 
                PCINT18, PCINT19, PCINT20, PCINT22, PCINT23, 
                %
                PCMSK0, PCMSK1, PCMSK2, PGERS, PGWRT, 
                %
                PINB, 
                PINB0, PINB1, PINB2, PINB3, PINB4, PINB5, PINB6,
                PINB7, 
                PINC, 
                PINC0, PINC1, PINC2, PINC3, PINC4, PINC5, PINC6,  
                PIND, 
                PIND0, PIND1, PIND2, PIND3, PIND4, PIND5, PIND6,
                PIND7, 
                %
                PORF, 
                PORTB, 
                PORTB0, PORTB1, PORTB2, PORTB3, PORTB4, PORTB5,
                PORTB6, PORTB7, 
                PORTC, 
                PORTC0, PORTC1, PORTC2, PORTC3, PORTC4, PORTC5,
                PORTC6, 
                PORTD, 
                PORTD0, PORTD1, PORTD2, PORTD3, PORTD4, PORTD5,
                PORTD6, PORTD7, 
                %
                PRADC, PRR, PRSPI, PRTIM0, PRTIM1, PRTIM2, 
                PRTWI, PRUSART0, PSRASY, PSRSYNC, PUD, PVINT21,
                PVINT5, 
                %
                REFS0, REFS1, RWWSB, RWWSRE, RXB80, RXC0, 
                RXCIE0, RXEN0, 
                %
                SE, SELFPRGEN, SM0, SM1, SM2, SMCR, SP0, SP1, 
                SP10, 
                SP2, SP3, SP4, SP5, SP6, SP7, SP8, SP9, 
                SPCR, SPDR, SPE, SPH, SPI2X, SPIE, SPIF, SPL,
                SPMCSR, SPMIE, SPR0, SPR1, SPSR, SREG, 
                %
                TCCR1A, TCCR1B, TCCR1C, TCCR2A, TCCR2B, TCN2UB,
                TCNT0, TCNT1, TCNT1H, TCNT1L, TCNT2, TCR2AUB, 
                TCR2BUB, TIFR0, TIFR1, TIFR2, TIMSK0, TIMSK1,
                TIMSK2, TOIE0, TOIE1, TOIE2, TOV0, TOV1, TOV2, 
                TSM, TTCR0A, TTCR0B, TWA0, TWA1, TWA2, TWA3, 
                TWA4, TWA5, TWA6, TWAM0, TWAM1, TWAM2, TWAM3, 
                TWAM4, TWAM5, TWAM6, TWAMR, TWAR, TWBR, TWCR, 
                TWDR, TWEA, TWEN, TWGCE, TWIE, TWINT, 
                TWPS0, TWPS1, TWS3, TWS4, TWS5, TWS6, TWS7, 
                TWSR, TWSTA, TWSTO, TWWC, TXB80, TXC0, TXCIE0, 
                TXEN0, 
                %
                U2X0, UBRR0, UBRR0H, UBRR0L, UCPHA0, UCPOL0, 
                UCSR0A, UCSR0B, UCSR0C, UCSZ00, UCSZ01, UCSZ02,
                UDORD0, UDR0, UDRE0, UMSEL00, UMSEL01, UPE0, 
                UPM00, UPM01, USBS0, 
                %
                WCOL, WDCE, WDE, WDIE, WDIF, WDP0, WDP1, WDP2, 
                WDP3, WDRF, WDTCSR, WGM00, WGM01, WGM02, WGM10,
                WGM11, WGM12, WGM13, WGM20, WGM21, WGM22, 
                %
                % Fuses
                %
                BODLEVEL0, BODLEVEL1, BODLEVEL2, 
                BOOTRST, BOOTSZ0, BOOTSZ1, 
                CKDIV8, CKOUT, 
                CKSEL0, CKSEL1, CKSEL2, CKSEL3, 
                DWEN, EESAVE, RSTDISBL, SPOIEN, 
                SUT0, SUT1, WDTON, 
                %
                % Lock Bits
                %
                BLB01, BLB02, BLB11, BLB12,
                LB1, LB2
                %
                % General purpose registers
                %
                R0,  R1,  R2,  R3,  R4,  R5,  R6,  R7,  R8, 
                R9,  R10, R11, R12, R13, R14, R15, R16, R17, 
                R18, R19, R20, R21, R22, R23, R24, R25, R26, 
                R27, R28, R29, R30, R31, 
                X, XH, XL, Y, YH, YL, Z, ZH, ZL,
                },  
% 
%
  %%% Keyword Color Group 3 %%%  (called KEYWORD1 by arduino)
  keywordstyle=[3]\color{arduinoOrange},
  keywords=[3]{  % Add AVR assembler directives as a 3rd group
                BYTE,
                %
                CSEG, CSEGSIZE,
                %
                DB, DD, DEF, DEVICE, DSEG, DQ, DW,
                %
                ELSE, ELIF, ENDIF, ENDM, ENDMACRO, EQU, 
                ERROR, ESEG, EXIT,
                %
                IF, IFDEF, IFNDEF, INCLUDE,
                %
                LIST, LISTMAC,
                %
                MACRO, MESSAGE,
                %
                NOLIST, NOOVERLAP,
                %
                ORG, OVERLAP,
                %
                SET,
                %
                UNDEF, 
                %
                % Add AVR-GAS assembler directives
                %
                ABORT, ALIGN, APP-FILE, ASCII, ASCIZ, 
                %
                BALIGN, BYTE, 
                %
                COMM, 
                %
                DATA, DEF, DESC, DIM, DOUBLE, 
                %
                EJECT, ELSE, ENDEF, ENDIF, EQU, EXTERN, 
                %
                FILE, FILL, FLOAT, 
                %
                GLOBAL, 
                %
                HWORD,
                %
                IDENT, IF, INCLUDE, INT, IRP, IRPC, 
                %
                LCOMM, LFLAGS, LINE, LN, LIST, 
                %
                MACRO,
                %
                NOLIST, 
                %
                OCTA, ORG, 
                %
                P2ALIGN, PSIZE, 
                %
                QUAD, 
                %
                REPT, 
                %
                SUBTTL, SCL, SECTION, SET, SHORT, 
                SINGLE, SHORT, SPACE, STABD, STABN, 
                STABS, STRING, 
                %
                TAG, TEXT, TITLE, TYPE, 
                %
                VAL, 
                %
                WORD,
                %
                % Add AVR-GCC preprocessor directives
                %
                DEFINE, INCLUDE, UNDEF, WARNING, PRAGMA, 
                }, 
%
%
  %%% Keyword Color Group 4 %%%  (called KEYWORD2 by arduino)
  keywordstyle=[4]\color{arduinoGreen},
  keywords=[4]{  % Add AVR assembler functions as a 4th group
                ABS, 
                BYTE2, BYTE3, BYTE4, 
                DEFINED,
                EXP2,
                FRAC, 
                HIGH, HWRD, 
                INT,
                LOW, LWRD, LOG2, 
                PAGE, 
                Q7, Q15, 
                STRLEN,
                %
                % And some macros from avrlib
                %
                __SFR_OFFSET, _SFR_IO_ADDR, _SFR_IO_REG_P,
                _SFR_MEM_ADDR, _SFR_ASM_COMPAT, _SFR_MEM8,
                _SFR_MEM16, _SFR_MEM32, _SFR_IO8, _SFR_IO16,
                _SFR_BYTE, _SFR_WORD, _SFR_DWORD
                %
                % And avr-gas functions.
                %
                HH8, HI8, HLO8, HHI8,
                LO8,
                PM, PM_HH8, PM_HI8, PM_LO8,
                },      
%
%
  %%% Set Other Colors %%%
  stringstyle=\color{arduinoDarkBlue},    
  commentstyle=\color{arduinoGrey},    
%          
%   
  %%%% Line Numbering %%%%
  numbers=left,                    
  numbersep=5pt,                   
  numberstyle=\color{arduinoGrey}, 
  showstringspaces=false
  showlines=true
  emptylines=1
  numbers=left   
  %stepnumber=2,                      % show every 2 line numbers
%
%
  %%%% Code Box Style %%%%
  breaklines=true,                    % wordwrapping
  tabsize=4,         
  basicstyle=\ttfamily  
}
                                         %%%
%%% in the preamble, before \begin{document}, in your latex file.              %%%
%%%                                                                            %%%
%%% In your document, place your arduino code between:                         %%%
%%%   \begin{lstlisting}[language=AVRAssembly]                                 %%%
%%% and:                                                                       %%%
%%%   \end{lstlisting}                                                         %%%
%%%                                                                            %%%
%%% Or create your own style to add non-built-in functions and variables.      %%%
%%% -------------------------------------------------------------------------- %%%
%%% For LyX projects:                                                          %%%
%%%                                                                            %%%
%%% In Lyx, go to Document->Settings and click on LaTeX Preamble.              %%%
%%%                                                                            %%%
%%% Add the following text to the existing preamble:                           %%%
%%%                                                                            %%%
%%% \lstdefinestyle{AVRasm}{                                                   %%%
%%%   language=AVRAssembly,                                                    %%%
%%%   %% Add other words needing highlighting below %%                         %%%
%%%   morekeywords=[1]{},        % [1] -> dark blue - opcodes                  %%%
%%%   morekeywords=[2]{},        % [2] -> blue - registers & bit names         %%%
%%%   morekeywords=[3]{},        % [3] -> orange - assembler directives        %%%
%%%   morekeywords=[4]{},        % [4] -> green - assembler functions          %%%
%%% }                                                                          %%%
%%%                                                                            %%%
%%% To highlight some AVR assembly code, right-click the program listing       %%%
%%% insert, and select settings. On the advanced tab, add the text:            %%%
%%%    style=AVRasm                                                            %%%
%%%                                                                            %%%
%%% To highlight AVR assembly code inserts as a child document, right-click    %%%
%%% the inserted file, choose settings. At the bottom of the dialogue, add:    %%%
%%%    style=AVRasm                                                            %%%
%%% in the "More parameter" editbox on the right side.                         %%%
 %%%%%%%%%%%%%%%%%%%%%%%%%%%%%%%%%%%%%%%%%%%%%%%%%%%%%%%%%%%%%%%%%%%%%%%%%%%%%%%%

\usepackage{color}
\usepackage{listings}
\usepackage{courier}

%%% Define Custom IDE Colors %%%
\definecolor{arduinoGreen}    {rgb} {0.17, 0.43, 0.01}
\definecolor{arduinoGrey}     {rgb} {0.47, 0.47, 0.33}
\definecolor{arduinoOrange}   {rgb} {0.8 , 0.4 , 0   }
\definecolor{arduinoBlue}     {rgb} {0.01, 0.61, 0.98}
\definecolor{arduinoDarkBlue} {rgb} {0.0 , 0.2 , 0.5 }

%%% Define AVR Assembly Language %%%
\lstdefinelanguage{AVRAssembly}{
  sensitive=false,    % Keywords etc are not case sensitive
  morecomment=[l]{;}, % Line comments begin with ';'
  morecomment=[l]{//}, % Line comments begin with ';'
  morestring=[b]",    % Strings are in double quotes
%
%
  %%% Keyword Color Group 1 %%%  (called KEYWORD3 by arduino)
  keywordstyle=\color{arduinoDarkBlue},
  morekeywords={   % Add assembly instructions to group 1
                ADC, ADD, ADIW, AND, ANDI, ASR,
                %
                BCLR, BLD, BRBC, BRBS, BRCC, BRCS, BREAK,
                BREQ, BRGE, BRHC, BRID, BRIE, BRLO, BRLT,
                BRMI, BRNE, BRPL, BRSH, BRTC, BRTS, BRVC, BRVS,
                BSET, BST,
                %
                CALL, CBI, CBR, CLC, CLH, CLI, CLN, CLR, CLS,
                CLT, CLV, CLZ, COM, CP, CPC, CPI, CPSE,
                %
                DEC, DES,
                %
                EICALL, EIJMP, ELPM, EOR,
                %
                FMUL, FMULS, FMULSU,
                %
                ICALL, IJMP, IN, INC,
                %
                JMP,
                %
                LAC, LAS, LAT, LD, LDD, LDI, LDS, LPM, LSL,
                LSR,
                %
                MOV, MOVW, MUL, MULS, MULSU,
                %
                NEG, NOP,
                %
                OR, ORI, OUT,
                %
                POP, PUSH,
                %
                RCALL, RET, RETI, RJMP, ROL, ROR,
                %
                SBC, SBCI, SBI, SBIC, SBIS, SBIW, SBR, SBRC,
                SBRS, SEC, SEH, SEI, SEN, SER, SES, SET, SEV,
                SEZ, SLEEP, SPM, ST, STD, STS, SUB, SUBI, SWAP,
                %
                TST,
                %
                WDR,
                %
                XCH,
                },
%
%
  %%% Keyword Color Group 2 %%%  (called LITERAL1 by arduino)
  keywordstyle=[2]\color{arduinoBlue},
  keywords=[2]{ % Add registers and bit names to 2nd group
                ACBG, ACD, ACI, ACIC, ACIE, ACIS0, ACIS1, ACME,
                ACO, ACSR, ADATE, ADC, ADC0D, ADC1D, ADC2D,
                ADC3D, ADC4D, ADC5D, ADCH, ADCL, ADCSRA, ADCSRB,
                ADCW, ADEN, ADIE, ADIF, ADLAR, ADMUX, ADPS0,
                ADPS1, ADPS2, ADSC, ADTS0, ADTS1, ADTS2, AIN0D,
                AIN1D, AS2, ASSR,
                %
                BLBSET, BODS, BODSE, BORF,
                %
                CLKPCE, CLKPR, CLKPS0, CLKPS1, CLKPS2, CLKPS3,
                COM0A0, COM0A1, COM0B0, COM0B1, COM1A0, COM1A1,
                COM1B0, COM1B1, COM2A0, COM2A1, COM2B0, COM2B1,
                CPHA, CPOL, CS00, CS01, CS02, CS10, CS11, CS12,
                CS20, CS21, CS22,
                %
                DDB0, DDB1, DDB2, DDB3, DDB4, DDB5, DDB6, DDB7,
                DDC0, DDC1, DDC2, DDC3, DDC4, DDC5, DDC6,
                DDD0, DDD1, DDD2, DDD3, DDD4, DDD5, DDD6, DDD7,
                DDRB, DDRC, DDRD, DIDR0, DIDR1, DOR0, DORD,
                %
                EEAR, EEARH, EEARL, EECR, EEDR, EEMPE, EEPE,
                EEPM0, EEPM1, EERE, EERIE, EICRA, EIFR, EIMSK,
                EXCLK, EXTRF, FE0,
                %
                FOC0A, FOC0B, FOC1A, FOC1B, FOC2A, FOC2B,
                %
                GPIOR0, GPIOR1, GPIOR2, GTCCR,
                %
                ICES1, ICF1, ICIE1, ICNC1, ICR1, ICR1H, ICR1L,
                IDRIE0, INT0, INT1, INTF0, INTF1, ISC00, ISC01,
                ISC10, ISC11, IVCE, IVSEL,
                %
                MCUCR, MCUSR, MPCM0, MSTR, MUX0, MUX1, MUX2,
                MUX3,
                %
                OCF0A, OCF0B, OCF1A, OCF1B, OCF2A, OCF2B,
                OCIE0A, OCIE0B, OCIE1A, OCIE1B, OCIE2A, OCIE2B,
                OCR0A, OCR0B, OCR1A, OCR1AH, OCR1AL, OCR1B,
                OCR1BH, OCR1BL, OCR2A, OCR2AUB, OCR2B, OCR2BUB,
                OSCCAL,
                %
                PCICR, PCIE0, PCIE1, PCIE2, PCIF0, PCIF1,
                PCIF2, PCIFR,
                %
                PCINT0, PCINT1, PCINT2, PCINT3, PCINT4, PCINT6,
                PCINT7, PCINT8, PCINT9, PCINT10, PCINT11,
                PCINT12, PCINT13, PCINT14, PCINT16, PCINT17,
                PCINT18, PCINT19, PCINT20, PCINT22, PCINT23,
                %
                PCMSK0, PCMSK1, PCMSK2, PGERS, PGWRT,
                %
                PINB,
                PINB0, PINB1, PINB2, PINB3, PINB4, PINB5, PINB6,
                PINB7,
                PINC,
                PINC0, PINC1, PINC2, PINC3, PINC4, PINC5, PINC6,
                PIND,
                PIND0, PIND1, PIND2, PIND3, PIND4, PIND5, PIND6,
                PIND7,
                %
                PORF,
                PORTB,
                PORTB0, PORTB1, PORTB2, PORTB3, PORTB4, PORTB5,
                PORTB6, PORTB7,
                PORTC,
                PORTC0, PORTC1, PORTC2, PORTC3, PORTC4, PORTC5,
                PORTC6,
                PORTD,
                PORTD0, PORTD1, PORTD2, PORTD3, PORTD4, PORTD5,
                PORTD6, PORTD7,
                %
                PRADC, PRR, PRSPI, PRTIM0, PRTIM1, PRTIM2,
                PRTWI, PRUSART0, PSRASY, PSRSYNC, PUD, PVINT21,
                PVINT5,
                %
                REFS0, REFS1, RWWSB, RWWSRE, RXB80, RXC0,
                RXCIE0, RXEN0,
                %
                SE, SELFPRGEN, SM0, SM1, SM2, SMCR, SP0, SP1,
                SP10,
                SP2, SP3, SP4, SP5, SP6, SP7, SP8, SP9,
                SPCR, SPDR, SPE, SPH, SPI2X, SPIE, SPIF, SPL,
                SPMCSR, SPMIE, SPR0, SPR1, SPSR, SREG,
                %
                TCCR1A, TCCR1B, TCCR1C, TCCR2A, TCCR2B, TCN2UB,
                TCNT0, TCNT1, TCNT1H, TCNT1L, TCNT2, TCR2AUB,
                TCR2BUB, TIFR0, TIFR1, TIFR2, TIMSK0, TIMSK1,
                TIMSK2, TOIE0, TOIE1, TOIE2, TOV0, TOV1, TOV2,
                TSM, TTCR0A, TTCR0B, TWA0, TWA1, TWA2, TWA3,
                TWA4, TWA5, TWA6, TWAM0, TWAM1, TWAM2, TWAM3,
                TWAM4, TWAM5, TWAM6, TWAMR, TWAR, TWBR, TWCR,
                TWDR, TWEA, TWEN, TWGCE, TWIE, TWINT,
                TWPS0, TWPS1, TWS3, TWS4, TWS5, TWS6, TWS7,
                TWSR, TWSTA, TWSTO, TWWC, TXB80, TXC0, TXCIE0,
                TXEN0,
                %
                U2X0, UBRR0, UBRR0H, UBRR0L, UCPHA0, UCPOL0,
                UCSR0A, UCSR0B, UCSR0C, UCSZ00, UCSZ01, UCSZ02,
                UDORD0, UDR0, UDRE0, UMSEL00, UMSEL01, UPE0,
                UPM00, UPM01, USBS0,
                %
                WCOL, WDCE, WDE, WDIE, WDIF, WDP0, WDP1, WDP2,
                WDP3, WDRF, WDTCSR, WGM00, WGM01, WGM02, WGM10,
                WGM11, WGM12, WGM13, WGM20, WGM21, WGM22,
                %
                % Fuses
                %
                BODLEVEL0, BODLEVEL1, BODLEVEL2,
                BOOTRST, BOOTSZ0, BOOTSZ1,
                CKDIV8, CKOUT,
                CKSEL0, CKSEL1, CKSEL2, CKSEL3,
                DWEN, EESAVE, RSTDISBL, SPOIEN,
                SUT0, SUT1, WDTON,
                %
                % Lock Bits
                %
                BLB01, BLB02, BLB11, BLB12,
                LB1, LB2
                %
                % General purpose registers
                %
                R0,  R1,  R2,  R3,  R4,  R5,  R6,  R7,  R8,
                R9,  R10, R11, R12, R13, R14, R15, R16, R17,
                R18, R19, R20, R21, R22, R23, R24, R25, R26,
                R27, R28, R29, R30, R31,
                X, XH, XL, Y, YH, YL, Z, ZH, ZL,
                },
%
%
  %%% Keyword Color Group 3 %%%  (called KEYWORD1 by arduino)
  keywordstyle=[3]\color{arduinoOrange},
  keywords=[3]{  % Add AVR assembler directives as a 3rd group
                BYTE,
                %
                CSEG, CSEGSIZE,
                %
                DB, DD, DEF, DEVICE, DSEG, DQ, DW,
                %
                ELSE, ELIF, ENDIF, ENDM, ENDMACRO, EQU,
                ERROR, ESEG, EXIT,
                %
                IF, IFDEF, IFNDEF, INCLUDE,
                %
                LIST, LISTMAC,
                %
                MACRO, MESSAGE,
                %
                NOLIST, NOOVERLAP,
                %
                ORG, OVERLAP,
                %
                SET,
                %
                UNDEF,
                %
                % Add AVR-GAS assembler directives
                %
                ABORT, ALIGN, APP-FILE, ASCII, ASCIZ,
                %
                BALIGN, BYTE,
                %
                COMM,
                %
                DATA, DEF, DESC, DIM, DOUBLE,
                %
                EJECT, ELSE, ENDEF, ENDIF, EQU, EXTERN,
                %
                FILE, FILL, FLOAT,
                %
                GLOBAL,
                %
                HWORD,
                %
                IDENT, IF, INCLUDE, INT, IRP, IRPC,
                %
                LCOMM, LFLAGS, LINE, LN, LIST, LONG,
                %
                MACRO,
                %
                NOLIST,
                %
                OCTA, ORG,
                %
                P2ALIGN, PSIZE,
                %
                QUAD,
                %
                REPT,
                %
                SUBTTL, SCL, SECTION, SET, SHORT,
                SINGLE, SHORT, SPACE, STABD, STABN,
                STABS, STRING,
                %
                TAG, TEXT, TITLE, TYPE,
                %
                VAL,
                %
                WORD,
                %
                % Add AVR-GCC preprocessor directives
                %
                DEFINE, INCLUDE, UNDEF, WARNING, PRAGMA,
                },
%
%
  %%% Keyword Color Group 4 %%%  (called KEYWORD2 by arduino)
  keywordstyle=[4]\color{arduinoGreen},
  keywords=[4]{  % Add AVR assembler functions as a 4th group
                ABS,
                BYTE2, BYTE3, BYTE4,
                DEFINED,
                EXP2,
                FRAC,
                HIGH, HWRD,
                INT,
                LOW, LWRD, LOG2,
                PAGE,
                Q7, Q15,
                STRLEN,
                %
                % And some macros from avrlib
                %
                __SFR_OFFSET, _SFR_IO_ADDR, _SFR_IO_REG_P,
                _SFR_MEM_ADDR, _SFR_ASM_COMPAT, _SFR_MEM8,
                _SFR_MEM16, _SFR_MEM32, _SFR_IO8, _SFR_IO16,
                _SFR_BYTE, _SFR_WORD, _SFR_DWORD
                %
                % And avr-gas functions.
                %
                HH8, HI8, HLO8, HHI8,
                LO8,
                PM, PM_HH8, PM_HI8, PM_LO8,
                },
%
%
  %%% Set Other Colors %%%
  stringstyle=\color{arduinoDarkBlue},
  commentstyle=\color{arduinoGrey},
%
%
  %%%% Line Numbering %%%%
  numbers=left,
  numbersep=5pt,
  numberstyle=\color{arduinoGrey},
  showstringspaces=false
  showlines=true
  emptylines=1
  numbers=left
  %stepnumber=2,                      % show every 2 line numbers
%
%
  %%%% Code Box Style %%%%
  breaklines=true,                    % wordwrapping
  tabsize=4,
  basicstyle=\ttfamily
}
